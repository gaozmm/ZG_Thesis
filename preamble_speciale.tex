% % Preamble (Martin Vindbæk Madsen (marts 2013))

% Størrelse papir, skriftstørrelse
\documentclass[a4paper,11pt,twoside,openright]{memoir} %\documentclass[a4paper,10pt,twoside,openright]{memoir}

% Opsætning af margin
\setlrmarginsandblock{*}{3.3cm}{1} %\setlrmarginsandblock{*}{4.0cm}{.8} 
\setulmarginsandblock{2.3cm}{*}{1}
\checkandfixthelayout

%%%%%%%%%%%%%%%%%%%
%-----ANVENDTE PAKKER---------start
%%%%%%%%%%%%%%%%%%%

% Danske bogstaver tilladt, orddeling mm.
\usepackage[utf8]{inputenc} % utf8,  ansinew, latin1
\usepackage[english,danish]{babel} % ordbog
\usepackage[T1]{fontenc} % bedre orddeling og ofte påkrævet af forskellige fonte
\usepackage{lmodern} % lmodern, palatino, mathpazo eller lignende
	 
	 
% Matematik, kemi, fysik  relaterede pakker samt symboler
\usepackage{amsmath,amssymb,stmaryrd,amsfonts,amssymb} % bedre matematik og ekstra fonte
\usepackage{mathptmx}
\usepackage[fixamsmath]{mathtools} %tillader brugen af \ArrowBetweenLines
	\providecommand{\abs}[1]{\lvert#1\rvert} % gør det muligt at bruge kommandoen \abs{arg1} til at skrive længden af en vektor.
\usepackage{xfrac}
	\everymath=\expandafter{\the \everymath \displaystyle} % forstørrer matricer og math generelt
\usepackage{icomma} %forhindrer latex i at lave et mellerum efter komma i math-mode...
%\usepackage[amssymb]{SIunits} % SI units i math-mode
\usepackage{siunitx}
	\DeclareSIUnit\century{century}
	\DeclareSIUnit\year{yr}

\usepackage[version=3]{mhchem} % praktisk til at skrive kemi inkl. reaktionsskemaer
\usepackage{rsphrase} % til R og S sætninger
\usepackage{latexsym}% symboler, saa skulle alle symboler i The Not So... virke
\usepackage{textcomp} % adgang til tekstsymboler
\usepackage[framed,numbered,autolinebreaks,useliterate]{mcode} %Be.nyttes til at lave MatLab lignende kode i LaTeX 
\usepackage[amsmath,thmmarks,framed]{ntheorem}
%\usepackage{ulem} % benyt \uuline{text} til matricer
\usepackage{mathrsfs} % til at lave krøllede bogstaver til baser. \mathscr

% Figurer, grafik og tabeller
\usepackage{multicol}
\usepackage{xcolor,colortbl}
\usepackage{array,longtable}
\usepackage{graphicx} % pakke til inklusion af grafik
\usepackage[percent]{overpic} % pakke som gør det nemt at overlappe billeder. Brug \begin{overpic} og \put
\usepackage{adjustbox} %use \begin{adjustbox}{center} <your table> \end{adjustbox} to adjust box on page.
\usepackage{lipsum}
\newenvironment{bottompar}{\par\vspace*{\fill}}{\clearpage}

\usepackage{tikz}
\usetikzlibrary{calc,arrows,circuits.ee.IEC,positioning,
	shapes.geometric,decorations.pathmorphing,shapes,automata,backgrounds,petri,fit}
%\def\xcolorversion{2.00}
%\def\xkeyvalversion{1.8}
%\usepackage[version=0.96]{pgf}
\usepackage{ctable}
\usepackage{pgfplots}
\usepackage{epstopdf}
\usepackage{pdfpages} % skriv \includepdf{filename} man behøver ikke .pdf
\usepackage{color,calc,soul}
\usepackage{comment}
\usepackage{tocloft}
\usepackage{shorttoc}
\usepackage{multirow}
\usepackage{framed} %bruges under teoremopsætning
\usepackage{capt-of}
\usepackage{enumitem}
\usepackage{wrapfig}
\usepackage{sidecap}
\usepackage{caption}
\usepackage{subcaption}
\usepackage{pdflscape}
\usepackage{rotating}

% Nummerering
\usepackage{chngcntr} %Gør det muligt at ændre nummerering af alt muligt

% Noter, krydshenvisninger mm.
\usepackage[danish]{varioref} % smarte krydsreferencer via \vref
%\usepackage{todonotes} 
\usepackage{url}
	\urlstyle{same}
%\usepackage[draft,notcite,notref]{showkeys} % labels i margin
%\usepackage{makeidx}
%\usepackage{showidx}
%	\showindexmarks 
%%	\hideindexmarks

	
% Litteraurhenvisninger
\usepackage[square,authoryear,super,comma,sort&compress]{natbib} % opsætning til bibtex, vedr. referencer
\bibliographystyle{unsrt} % achemso plainnat apsrev4-1      husk ingen endelser .bst

% Hyperref
\usepackage[colorlinks=true, linkcolor=black,citecolor=black,urlcolor=black]{hyperref} % Laver links klickbare, og blå. SKAL stå som den sidste pakke der hentes!!!!
\usepackage{memhfixc}
%%%%%%%%%%%%%%%%%%%
%-----ANVENDTE PAKKER---------slut
%%%%%%%%%%%%%%%%%%%


%%%%%%%%%%%%%%%%%%%
% -------OPSÆTNINGER-------- start
%%%%%%%%%%%%%%%%%%%


\pgfplotsset{compat=1.3}


% Litteraturhenvisning
\def\bibfont{\footnotesize} %ændrer størrelsen på skriften i bibliography. Man kan også vælge \small eller andre størrelser. Kan kun benyttes ved natbib pakken.




%%%%%%%%%%%%%%%%%%%
% -----ToC - opsætning----------start
%%%%%%%%%%%%%%%%%%%
\setsecnumdepth{subsection} % eller hvor dybt man nu ønsker at have opskrifterne nummereret
\maxsecnumdepth{subsection}
\settocdepth{subsection} % hvor dybt ned vi ønsker ting med i indholdsfortegnelsen

% de næste linier forårsager, at der skrives chapter 1 og Appendix A i ToC'en
\renewcommand\cftchaptername{\chaptername~}
\renewcommand\cftappendixname{\appendixname~}
\renewcommand*{\cftchapteraftersnum}{.} % dot after number

%fjerner alle prikker mellem overskrift og sidetal
%\renewcommand{\cftdotsep}{\cftnodots}

%den følgende kan ændre vertikale afstande og derved bruges til at justere hvor sideskiftet kommer til at ligge
\renewcommand{\cftbeforesectionskip}{0.2em}
\renewcommand{\cftbeforesubsectionskip}{0.1em}

%De næste 6 kommandoer rykker sectionsindgangene i Toc'en helt til venstre og sørger for at resten følger med.
\setlength{\cftsectionindent}{0cm}
\setlength{\cftsubsectionindent}{1.2cm}
\setlength{\cftsubsubsectionindent}{2.4cm}

\setlength{\cftsectionnumwidth}{2.2em}
\setlength{\cftsubsectionnumwidth}{3em}
\setlength{\cftsubsubsectionnumwidth}{4.7em}

%%%%%%%%%%%%%%%%%%%
% -----ToC - opsætning----------slut
%%%%%%%%%%%%%%%%%%%

% Lav index
\makeindex


%%%%%%%%%%%%%%%%%%%
%-------Chapterstyle (daleif1)--------start
%%%%%%%%%%%%%%%%%%%

%\usepackage{color,calc,graphicx,soul,lmodern}
\definecolor{niceblue}{rgb}{.29,.29,.29}
\definecolor{nicered}{rgb}{.647,.129,.149}
\makeatletter
\newlength\dlf@normtxtw
\setlength\dlf@normtxtw{\textwidth}
\def\myhelvetfont{\def\sfdefault{mdput}}
\newsavebox{\feline@chapter}
\newcommand\feline@chapter@marker[1][4cm]{%
\sbox\feline@chapter{%
\resizebox{!}{#1}{\fboxsep=1pt%
\colorbox{niceblue}{\color{white}\bfseries\thechapter}%
}}%
\rotatebox{90}{%
\resizebox{%
\heightof{\usebox{\feline@chapter}}+\depthof{\usebox{\feline@chapter}}}%
{!}{\scshape\so\@chapapp}}\quad%
\raisebox{\depthof{\usebox{\feline@chapter}}}{\usebox{\feline@chapter}}%
}
\newcommand\feline@chm[1][4cm]{%
\sbox\feline@chapter{\feline@chapter@marker[#1]}%
\makebox[0pt][l]{% aka \rlap
\makebox[1cm][r]{\usebox\feline@chapter}%
}}
\makechapterstyle{daleif1}{
\setlength\beforechapskip{-\topskip-\baselineskip} % denne linje får kapitlerne til at starte højere oppe på siderne...
\renewcommand\chapnamefont{\normalfont\Large\scshape\raggedleft\so}
\renewcommand\chaptitlefont{\normalfont\huge\bfseries}%\color{ForestGreen}} %stod der før, men den meldte fejl pga bfseries: \renewcommand\chaptitlefont{\normalfont\huge\bfseries\scshape}
\renewcommand\chapternamenum{}
\renewcommand\printchaptername{}
\renewcommand\printchapternum{\null\hfill\feline@chm[2.5cm]\par}
\renewcommand\afterchapternum{\par\vskip\midchapskip}
% jeg ændrede linien lige under til linien under igen for at flytte kapitelnavnene over til venstre.
%\renewcommand\printchaptertitle[1]{\chaptitlefont\raggedleft ##1\par}
\renewcommand\printchaptertitle[1]{\chaptitlefont\raggedright ##1\par}
}
\makeatother
\chapterstyle{daleif1}

%%%%%%%%%%%%%%%%%%%
%-------Chapterstyle (daleif1)--------slut
%%%%%%%%%%%%%%%%%%%


%%%%%%%%%%%%%%%%%%%
%-------Appendix på dansk---------------start
%%%%%%%%%%%%%%%%%%%

%\addtocaptionsdanish{%
%%\renewcommand{\appendixpagename{Appendiks}}
%\renewcommand{\appendixname{Appendiks}}
%}

%\renewcommand\cftappendixname {\appendixname~}
%\renewcommand\appendixpagename {Appendiks}
%\renewcommand\appendixtocname {Appendiks}
%\addto\captionsdanish{
%\renewcommand\appendixname{Appendiks}
%}

%%%%%%%%%%%%%%%%%%%
%-------Appendix  på dansk---------------slut
%%%%%%%%%%%%%%%%%%%


%%%%%%%%%%%%%%%%%%%
%-------Pagestyle---------------start
%%%%%%%%%%%%%%%%%%%

% % Bris - pagestyle: 
\makepagestyle{bris}
\makeoddfoot{bris}{}{}{\thepage}
\makeevenfoot{bris}{\thepage}{}{}
\makepsmarks{bris}{%
\def\chaptermark##1{%
\markboth{\chaptername\ \thechapter.\ ##1}{}%
}
\def\sectionmark##1{%
\markright{\thesection. ##1}%
}
\def\subsectionmark##1{%
\markright{##1}}
}
\pagestyle{bris}

% % This - pagestyle: 
\makepagestyle{this}
\makeevenhead{this}{}{}{}
\makeoddhead{this}{}{}{}
\makeevenfoot{this}{\thepage}{\small{\RapportTitelFooter}}{}
\makeoddfoot{this}{}{\small{\RapportTitelFooter}}{\thepage}
\makefootrule{this}{\textwidth}{0.4pt}{3pt}
\makepsmarks{this}{%
\def\chaptermark##1{%
\markboth{\chaptername\ \thechapter.\ ##1}{}%
}
\def\sectionmark##1{%
\markright{\thesection. ##1}%
}
\def\subsectionmark##1{%
\markright{##1}}
}
\pagestyle{this}

% % dktug1 - pagestyle
\makepagestyle{dktug1}
\makeevenhead{dktug1}{\footnotesize\scshape\leftmark}{}{}
\makeoddhead{dktug1}{}{}{\footnotesize\scshape\rightmark}
\makeevenfoot{dktug1}{\thepage}{\small{\docFooter}}{}
\makeoddfoot{dktug1}{}{\small{\docFooter}}{\thepage}
\makeheadrule{dktug1}{\textwidth}{0.4pt}
\makefootrule{dktug1}{\textwidth}{0.4pt}{3pt}
\makepsmarks{dktug1}{%
\def\chaptermark##1{%
\markboth{\chaptername\ \thechapter.\ ##1}{}%
}
\def\sectionmark##1{%
\markright{\thesection. ##1}%
}
\def\subsectionmark##1{%
\markright{##1}}
}
\pagestyle{dktug1}


%fjerner sidetal fra midten af \appendixpage
\copypagestyle{part}{empty}
%\makeevenhead{plain}{}{}{}
%\makeoddhead{plain}{}{}{}

%flytter sidetal fra midten af første side i indholdsfortegnelsen
\copypagestyle{plain}{bris}
%\makeevenhead{plain}{}{}{}
%\makeoddhead{plain}{}{}{}

%%%%%%%%%%%%%%%%%%%
%-------Pagestyle---------------slut
%%%%%%%%%%%%%%%%%%%


%%%%%%%%%%%%%%%%%%%%%%%%%%%%%%%%%%%%%%%%%%%%%%%%%%%%%%%%%%%%
%%%%%%% Opsætning af gråtonede environments (start) %%%%%%%%
%%%%%%%%%%%%%%%%%%%%%%%%%%%%%%%%%%%%%%%%%%%%%%%%%%%%%%%%%%%%
%\definecolor{shadecolor}{gray}{0.75}
%\theoremheaderfont{\large\bfseries\color{niceblue}}
%\theorembodyfont{\normalfont} \theoremseparator{.}
%\newtheorem{innerthm}{Summary}[chapter]
%\newenvironment{thm}{%
%\begin{shaded}
%\setlength\theorempreskipamount{0pt}
%\setlength\theorempostskipamount{0pt}
%\begin{innerthm}%
%}{\end{innerthm}\end{shaded}}
%%%%%%%%%%%%%%%%%%%%%%%%%%%%%%%%%%%%%%%%%%%%%%%%%%%%%%%%%%%%
%%%%%%% Opsætning af gråtonede environments (slut) %%%%%%%%
%%%%%%%%%%%%%%%%%%%%%%%%%%%%%%%%%%%%%%%%%%%%%%%%%%%%%%%%%%%%



%%%%%%%%%%%%%%%%%%%%%%%%%%%%%%%%%%%%%%%%%%%%%%%%%%%%%%%%%%%%
%%%%%%%%%%%      egne kommandoer (start)    %%%%%%%%%%%%%%%%
%%%%%%%%%%%%%%%%%%%%%%%%%%%%%%%%%%%%%%%%%%%%%%%%%%%%%%%%%%%%
\newcommand{\M}{\textsc{m}}
\newcommand\rt{\mbox{room temperature}}
\newcommand\halv{\frac{1}{2}}
\newcommand{\deltao}{$\delta^{18}\mathrm{O}$ }
\newcommand{\deltad}{$\delta \mathrm{D}$ }

\newcommand{\HHO}[1]{$ \mathrm{H}_2^{#1}\mathrm{O} $}
\newcommand{\HDO}{$ \mathrm{HD}^{16}\mathrm{O} $}


%\renewcommand{\eqref}[1]{Eq. \eqref{#1}}

\newcommand{\tabref}[1]{Table \ref{#1}}

\newcommand{\figref}[1]{Figure \ref{#1}}
%%%%%%%%%%%%%%%%%%%%%%%%%%%%%%%%%%%%%%%%%%%%%%%%%%%%%%%%%%%%
%%%%%%%%%%%      egne kommandoer (start)    %%%%%%%%%%%%%%%%
%%%%%%%%%%%%%%%%%%%%%%%%%%%%%%%%%%%%%%%%%%%%%%%%%%%%%%%%%%%%


%%%%%%%%%%%%%%%%%%%%%%%%%%%%%%%%%%%%%%%%%%%%%%%%%%%%%%%%%%%%
%%%%%%   Konstruktion af Scheme-environment (start)  %%%%%%%
%%%%%%%%%%%%%%%%%%%%%%%%%%%%%%%%%%%%%%%%%%%%%%%%%%%%%%%%%%%%
\newcommand{\schemename}{Skema}
\newcommand{\listschemename}{}
\newlistof{listofschemes}{sch}{\listschemename}
\newfloat[chapter]{skema}{sch}{\schemename}
\newfixedcaption{\fschecaption}{skema}
\newlistentry{skema}{sch}{0}
%%%%%%%%%%%%%%%%%%%%%%%%%%%%%%%%%%%%%%%%%%%%%%%%%%%%%%%%%%%%
%%%%%%   Konstruktion af Scheme-environment (slut)  %%%%%%%%
%%%%%%%%%%%%%%%%%%%%%%%%%%%%%%%%%%%%%%%%%%%%%%%%%%%%%%%%%%%%


%%%%%%%%%%%%%%%%%%%%%%%%%%%%%%%%%%%%%%%%%%%%%%%%%%%%%%%%%%%%
%%%%%%%%%%%%   Opsætning af captions (start)  %%%%%%%%%%%%%%
%%%%%%%%%%%%%%%%%%%%%%%%%%%%%%%%%%%%%%%%%%%%%%%%%%%%%%%%%%%%
% pakken ccaption er en integreret del af memoir-klassen og dette åbner muligheden for brug af følgende%
\captiondelim{: }
\captionnamefont{\itshape\bfseries\small}%\color{ForestGreen}}
\captiontitlefont{\small}
%\belowcaptionskip=0pt
\abovecaptionskip=4pt
\setlength{\belowcaptionskip}{4pt}

%%%%%%%%%%%%%%%%%%%%%%%%%%%%%%%%%%%%%%%%%%%%%%%%%%%%%%%%%%%%
%%%%%%%%%%%%   Opsætning af captions (slut)  %%%%%%%%%%%%%%
%%%%%%%%%%%%%%%%%%%%%%%%%%%%%%%%%%%%%%%%%%%%%%%%%%%%%%%%%%%%

%%%%%%%%%%%%%%%%%%%%%%%%%%%%%%%%%%%%%%%%%%%%%%%%%%%%%%%%%%%%
%%%%%%%%%%%%   Tikz environments (start)  %%%%%%%%%%%%%%
%%%%%%%%%%%%%%%%%%%%%%%%%%%%%%%%%%%%%%%%%%%%%%%%%%%%%%%%%%%%
\tikzset{circuit declare symbol=relay,
	circuit ee IEC/.append style=
	{
		set relay graphic = relay IEC graphic
	},
	relay IEC graphic/.style={
		circuit symbol open,
		circuit symbol size=width 1 height 2.25,
		shape=generic relay IEC,
		transform shape
	}
}

\makeatletter

\pgfdeclareshape{generic relay IEC}{
	\inheritsavedanchors[from=rectangle ee]
	\inheritanchor[from=rectangle ee]{center}
	\inheritanchor[from=rectangle ee]{north}
	\inheritanchor[from=rectangle ee]{south}
	\inheritanchor[from=rectangle ee]{east}
	\inheritanchor[from=rectangle ee]{west}
	\inheritanchor[from=rectangle ee]{north east}
	\inheritanchor[from=rectangle ee]{north west}
	\inheritanchor[from=rectangle ee]{south east}
	\inheritanchor[from=rectangle ee]{south west}
	\inheritanchor[from=rectangle ee]{input}
	\inheritanchor[from=rectangle ee]{output}
	\inheritanchorborder[from=rectangle ee]
	
	\backgroundpath{
		% Corners
		\pgf@process{\pgfpointadd{\southwest}{\pgfpoint{\pgfkeysvalueof{/pgf/outer xsep}}{\pgfkeysvalueof{/pgf/outer ysep}}}}
		\pgf@xa=\pgf@x \pgf@ya=\pgf@y
		\pgf@process{\pgfpointadd{\northeast}{\pgfpointscale{-1}{\pgfpoint{\pgfkeysvalueof{/pgf/outer xsep}}{\pgfkeysvalueof{/pgf/outer ysep}}}}}
		\pgf@xb=\pgf@x \pgf@yb=\pgf@y
		% Start point
		\pgfpathmoveto{\pgfqpoint{\pgf@xa}{\pgf@ya}}
		% Lines
		\pgfpathlineto{\pgfqpoint{\pgf@xa}{\pgf@yb}}
		\pgfpathlineto{\pgfqpoint{\pgf@xb}{\pgf@yb}}
		\pgfpathlineto{\pgfqpoint{\pgf@xb}{\pgf@ya}}
		\pgfpathclose
		% Diagonal line
		% pgf@x[ab] contain the distance to the line.
		% Half the length of the line plus this (X) distance
		% will yield a good (Y) coordinate.
		% This is slightly hackish...
		\pgfpathmoveto{\pgfqpoint{\pgf@xa}{\pgf@xb}}
		\pgfpathlineto{\pgfqpoint{\pgf@xb}{\pgf@xa}}
	}
}

\makeatother

\newcommand{\twovalve}[3]{%
	\draw[fill=black,rotate around={#3:(#1,#2)}] (#1-.1,#2) -- (#1+.1,#2) -- (#1-.1,#2-.4) -- (#1+.1,#2-.4) -- cycle ;
	%\draw (#1,#2-1) -- (#1-.5,#2-1);
	%\draw (#1-.5,#2-1.45) rectangle (#1-1.25,#2-.65);
	%\node at (#1-2.5,#2-1) { #3};
}

\newcommand{\threevalve}[3]{%
	\draw[fill=black,rotate around={#3:(#1,#2)}] (#1-.1,#2) -- (#1+.1,#2) -- (#1-.1,#2-.6) -- (#1+.1,#2-.6) -- cycle ;
	\draw[fill=black,rotate around={#3:(#1,#2)}] (#1-0.3,#2-.2) --  (#1-0.015,#2-0.3) -- (#1-0.3,#2-.4) --  cycle ;
	%\draw  (#1-0.3,#2-.2) circle (0.05cm);
}

\newcommand{\valveright}[3]{%
	\draw (#1,#2+.3) --  (#1+0.7,#2) -- (#1,#2-.3) --  cycle ;
}

\newcommand{\process}[2]{%
	\draw [rounded corners=1.5mm] (#1,#2) rectangle (#1+1.5,#2+.5);
}

\tikzset{snake it/.style={decorate, decoration=snake}}

%%%%%%%%%%%%%%%%%%%%%%%%%%%%%%%%%%%%%%%%%%%%%%%%%%%%%%%%%%%%
%%%%%%%%%%%%   Tikz environments (start)  %%%%%%%%%%%%%%
%%%%%%%%%%%%%%%%%%%%%%%%%%%%%%%%%%%%%%%%%%%%%%%%%%%%%%%%%%%%


%\includeonly{ResultatDiskussion} %man skriver bare "," og så den næste man vil inkludere. ResultatDiskussion,Experimental,


%%%%%%%%%%%%%%%%%%%%%%%%%%%%%%%%%%%%%%%%%%%%%%%%%%%%%%%%%%%%
%%%%%%%%%%
% Change the depth of the Table of content (TOC)
\setcounter{tocdepth}{3}
\setcounter{secnumdepth}{2}

% Change the numbering strategy in figure
\counterwithout{equation}{chapter}
\counterwithin{equation}{section}
\counterwithout{figure}{chapter}
\counterwithin{figure}{section} 

\makeatletter
\@removefromreset{figure}{chapter}
\@addtoreset{figure}{section}
\renewcommand{\thefigure}{\thesection.\@arabic\c@figure}
\makeatother